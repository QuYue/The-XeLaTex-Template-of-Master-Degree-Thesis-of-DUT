% !TEX TS-program = XeLaTeX
% !TEX encoding = UTF-8 Unicode

\chapter{绪论}
\label{chap01}
\section{研究背景}
为了方便在overleaf上写毕业论文,于是我在之前的模板的基础上制作本模板

这里我就写一下需要的配置好了。
为了方便在overleaf以及任意环境下使用,
已经提前下载好了必要的字体于"fonts"文件夹中。

在Overleaf中,需要配置的环境如下(已经默认配置好了,大家不要必就行):
\begin{enumerate}
\item 编译器 Compiler:XeLaTex
\item TexLive版本 Tex Live Version: 2022
\item 主文件 Main document: main.tex
\end{enumerate}

当然也可以下载至本地的环境来编译。


\subsection{绪论(或引言)内容要求}
以下给出研究生院对引言内容的要求,格式的要求已经嵌入到本模版中:
\begin{enumerate}
\item 绪论(或引言)包含的内容有说明论文的主题和选题的范围、对本论文研究主要范围内已有文献的评述以及
  说明本论文所要解决的问题;
\item 注意不要与摘要内容雷同;
\item 建议与相关历史回顾、前人工作的文献评论、理论分析等相结合,如果引言部分省略,
  该部分内容在正文中单独成章,标题改为绪论,用足够的文字叙述。
\end{enumerate}

\zhangsan
{特别注意:是否如实引用前人结果反映的是学术道德问题,应明确写出同行相近的和已取得的成果,避免抄袭之嫌。} 

\section{国内外研究现状}
\textbf{xx的研究}

...

\textbf{xx的研究}

...




\section{研究创新}

本文对于xx的问题,提出了xx算法,xx。本文的贡献如下:
% 贡献
\begin{itemize}
    \item 贡献1。
    \item 贡献2。
    \item 贡献3。
\end{itemize}

\section{本文的组织结构}
本文主要围绕着xx开展,并辅以模拟数据和真实数据的实验,来验证提出算法的合理性和有效性。

第一章主要针对文章的背景进行介绍,并对国内外已有的研究进行了总结,接着提出本文的创新点,最后展示文章的整体结构。

第二章主要介绍了xx...。

第三章主要介绍了xxxx。

在最后,本文会对当前的研究加以总结,指出算法的局限性,并对其后续的研究及改进方向提出了个人的建议。
