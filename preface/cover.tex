% !TEX TS-program = XeLaTeX
% !TEX encoding = UTF-8 Unicode

%%%%%%%%%%%%%%%%%%%%%%%%%%%%%%%%%%%%%%%%%%%%%%%%%%%%%%%%%%%%%%%%%%%%%% 
% 
%	大连理工大学硕士论文 XeLaTeX 模版 —— 封面文件 cover.tex
%	版本:1.0
%	最后更新:2023.01.07
%   修改者:QuYue (E-mail: quyue1541@gmail.com)
%   编译环境:Overleaf + TeXLive 2022
%	原修改者:Yuri (E-mail: yuri_1985@163.com)
% 
%%%%%%%%%%%%%%%%%%%%%%%%%%%%%%%%%%%%%%%%%%%%%%%%%%%%%%%%%%%%%%%%%%%%%% 
\cdegree{硕~~士~~学~~位~~论~~文}
\ctitle{大连理工大学硕士学位论文模版}
\etitle{The \XeLaTeX{} Template of Master Degree Thesis  of DUT}

% 根据需要添加字符间距
\csubject{{\quad\;}~计算机科学与技术}
\cauthor{{\quad\;}~~~~~~~~~~~~~张三}
\cauthorno{{\quad\;}~~~~~~~~   20000000}
\csupervisor{{\quad\;}~~~~~~王老五~~~~教授 }

% 这里默认使用最后编译的时间,也可自行给定日期,注意汉字和数字之间的空格。
\cdate{{\quad\;}~~~\the\year~年~\the\month~月~\the\day~日}

\cabstract{
  本模版是根据大连理工大学硕士学位论文格式规范制作的硕士学位论文模板。
  
  本模板是基于北京大学、清华大学、哈尔滨工业大学等高校的硕博士论文模板,
  并按照大连理工大学硕士学位论文格式规范开发的论文模板,
  经过多人完善和修改,目前已经基本满足了论文规范的要求,
  而且易用性良好,功能强大。不过,可能还存在着一些问题,
  欢迎大家积极使用本模版,反馈遇到的问题,以便不断对其进行改进。
  
  当然这个模板仅仅是一个开始,希望有更多的人能够参与进来,
  不断改进准确性、易用性和较好的可维护性,造福需要的兄弟姐妹们。
  总体上来说,当前这个模板还是很值得推荐使用的。
  
  本模板的目的旨在推广这一优秀的排版软件在大工的应用,
  为广大同学提供一个方便、美观的论文模板,减少论文撰写格式方面的麻烦。

  和过去的版本不同的是,本版本的模板基本解决了过去版本存在相关的问题,并可以直接在overleaf上进行编辑,
  更为方便。
  
  以下顺便补充一些研究生院所提供的~Word~模版中的注意事项
  (略去已经嵌入到此模版中的内容):
    \begin{asparaenum}
  \item 论文摘要是学位论文的缩影,文字要简练、明确。内容要包括目的、方法、结果和结论。
    单位制一律换算成国际标准计量单位制,除特别情况外,数字一律用阿拉伯数码。
    文中不允许出现插图。重要的表格可以写入;
  \item 篇幅以一页为限,字数为~600-800~字
    (工程硕士、MBA、EMBA、MPA~等专业学位论文字数为~400-500~字);
  \item 摘要正文后,列出~3-5~个关键词。
    关键词请尽量用《汉语主题词表》等词表提供的规范词。
    关键词词间用分号间隔,末尾不加标点,3-5~个。
  \end{asparaenum}
}


\ckeywords{写作规范;排版格式;硕士学位论文;模版}

\eabstract{
  This is a template of master degree thesis of Dalian University of Technology,
  which is built according to the required format.
  
  内容应与“中文摘要”对应。使用第三人称,最好采用现在时态编写。
}

\ekeywords{Write Criterion; Typeset Format; Master's Degree Paper; Template}

\makecover 